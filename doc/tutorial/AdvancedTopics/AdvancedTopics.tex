
\begin{htmlonly}
\input{../../htmlinc}
\end{htmlonly}


%=============================================================
\startdocument
\chapter{Advanced Topics}
\label{AdvancedTopics}
%=============================================================

\section{Introduction}

After having read this chapter you will be familiar with:

\begin{itemize}
\item The Architecture of COVISE
\item how to prepare COVISE for distributed and collaborative working
\item how to run COVISE across firewalls
\item how to include a remote hosts or partner into a session
\item Starting modules on remote computers
\item Multiuser Sessions
\end{itemize}

In COVISE it is possible to run modules on remote computers. This is also known as 
"Distributed Computing". By distributing modules across a network one can make use 
of remote resources for example of a compute server with more CPUs or memory than on a 
local workstation or PC. The COVISE session is controlled from the Mapeditor on the
local workstation. Remote hosts are included in the session via the menu item 
CSCW \latexonly $>>$ \endlatexonly \begin{htmlonly} > > \end{htmlonly} AddHost (CSCW = \underline{C}omputer
\underline{S}upported \underline{C}ollaborative \underline{W}ork).

In a multiuser session each participant has it's own Mapeditor and Renderer. The 
session has to be initiated by one partner who adds the other partners to the 
session (menu item CSCW \latexonly $>>$ \endlatexonly \begin{htmlonly}> > \end{htmlonly} AddPartner). The initiating partner plays the master role, 
which means that he has the control over the Mapeditor and the Renderer. If he e.g. 
changes the camera position in the Renderer all other partner's cameras are synchronised 
with the master camera. The master role can be exchanged between partners. This way of 
working together in a multiuser session is also known as "Collaborative Working" where 
COVISE is regarded as a "Shared Application" which is aware of the sharing.

As the hosts of the partners can also be used for distributed computing COVISE extends 
far beyond a "Shared Application" such as the ones based on X Windows sharing or a 
Windows application shared via Netmeeting.



\section{Architecture}

This section provides background information on the COVISE architecture and explains how distributed and
collaborative working is implemented.

\vspace{0.5cm}
\begin{Large}{\bf Distributed Working}\end{Large}
\vspace{0.5cm}

\begin{covimg}{AdvancedTopics}{DistributedWorking}{Distributed Working}{0.7}\end{covimg}
\begin{htmlonly}
Figure 5.1: Distributed Working
\vspace{0.5cm}
\end{htmlonly}


Figure 5.1 shows the elements of distributed working in COVISE. The application consists 
of three modules: a module which reads in data (READ) a module which extracts a special 
feature (FILTER) a module which displays the extracted data (RENDER). As the filter 
module consumes much CPU time and memory it will be started on a remote compute server. 

The first process which is started when covise is typed in is the Controller which in 
turn starts the user interface process (Mapeditor) and the data management process (CRB). 
As soon as another host is included in the session a CRB is started on that computer. 
The read module is started on the local workstation, the filter module on the remote 
computer (see section "Starting a Module on a Remote Computer" later in this chapter) 
and the renderer on the local workstation. The black arrows between the processes 
Controller, UIF, CRB and the modules indicate TCP sockets, the blue arrows indicate 
shared memory access. 

When the user executes the pipeline the Controller sends a start message to the read 
module. The read module reads in the data file and creates a COVISE data object (1) in 
shared memory and after processing tells the Controller that it has finished. The 
Controller informs the filter module on the remote computer to start. The filter 
module asks its data management process (CRB) for the data object (1). As this CRB 
doesn't have the object it asks the CRB on the workstation for the object and copies 
it to it's shared memory (1'). The filter module now reads that data object,
computes something and puts the data object (2) into shared memory. It then tells 
the Controller that it has finished. The controller informs the Renderer module to start.
The Renderer asks the CRB for object (2) and as this object is not
available on the local workstations the CRB transfers it from the compute server into the shared memory of the
workstation. Now the renderer can access this object and display the data. 

\vspace{0.5cm}
\begin{Large}{\bf Collaborative Working}\end{Large}
\vspace{0.5cm}

In a collaborative session (Figure 5.2) a user interface process (Mapeditor) 
and a Renderer are started also on the remote machine. The Renderer module is the 
only module which is started on all computers in a session. 


\begin{covimg}{AdvancedTopics}{CollaborativeWorking}{Collaborative Working}{0.7}\end{covimg}
\begin{htmlonly}
Figure 5.2: Collaborative Working
\vspace{1cm}
\end{htmlonly}

\clearpage
\section{Preparing COVISE for distributed and Collaborative Working}


Every computer that will participate in a distributed or collaborative session should be included in the section
HostConfig in the file covise.config. For each host you have to specify the memory model for data exchange between
modules on the local machine, the execution mode and a timeout for TCP connections. 

\begin{verbatim}
HostConfig
{
    # Hostname MemoryModel ExecutionMode Timeout
    vista       shm        rexec       30
    visit       shm        rexec       30
}
\end{verbatim}

For workstations and PCs the memory model is shm which stands for shared memory.

The execution mode specifies the command which should be used to start the CRB on the remote computer. Possible
execution modes are:

\begin{itemize}
\item rexec
\item rsh
\item ssh
\item nqs
\item manual
\end{itemize}

For all execution modes besides manual one needs access to the account on the remote 
computer. For rexec one has to enter the hostname, the user id and the password on the 
remote machine (similar to logging in on the remote computer using telnet). rsh and ssh 
can only be used if they allow to log in without password specification (see man rsh and 
ssh for the files where allowed users are specified). nqs is not recommended, it can be 
used to put the CRB into a batch queue. Manual means that one has to start the CRB 
process manually on the remote machine. This can be useful for sessions across a 
firewall or if one doesn't have access to the remote account. In this case COVISE 
shows a command in the window where COVISE was started.

The time-out value specifies how many seconds a process will wait to be contacted by a 
new process that he initiated (e.g. the Controller waiting for a module). The default 
value is 5 seconds. For slow networks a time-out of 30 seconds is useful. For very 
slow networks even a higher value is recommended.


\section{COVISE across Firewalls}

As shown in Figure 5.2 COVISE uses TCP sockets for communication with remote hosts. 
A socket is defined by an IP address, a port number and the protocol (here tcp). COVISE 
port numbers start by default at 31000. One can configure the start number in the 
file covise.config using the keyword COVISE\_PORT in the section network:
 
\begin{verbatim}
Network
{
    COVISE_PORT     5000
}
\end{verbatim}

For collaborative or distributed sessions across firewalls the firewall has to allow 
tcp connections to ports in both directions starting with the number defined in 
covise.config. You need as many ports as modules started during the whole session +3 
for distributed sessions or + 4 for collaborative sessions (if you load several maps 
in a session each map needs new ports). Depending on the execution mode the ports for 
rexec, rsh or ssh have to be allowed. For the execution mode manual no extra port 
is required.

Note:

If you use IP forwarding from your firewall to your local computer you
have to make additional configurations.
Every host that wants to connect to your session has to know that you
are behind a firewall and use IP forwarding.
Therefore you can tell COVISE not to connect to your machine but to your
firewall instead. This is done by adding an IP\_ALIAS entry on every client side. 
Assume your IP is 192.168.0.15 and your firewall has the IP 133.168.226.234 
from the outside. Then you have to add

\begin{verbatim}
Network {
   IP_ALIAS    192.168.0.15    133.168.226.234
   #           <your IP>       <your firewall IP> =

}
\end{verbatim}

to the config file on every host you want to connect to.


\section{Including a remote host or partner in the session}

\begin{covimg}{AdvancedTopics}{HostsMenu}{Hosts(CSCW) Menu}{0.7}\end{covimg}
\begin{htmlonly}
Figure 5.3: Hosts(CSCW) Menu
\vspace{0.5cm}
\end{htmlonly}

Figure 5.3 shows the menu item for adding a remote host or including a new partner 
into the session (CSCW = \underline{C}omputer \underline{S}upported \underline{C}ollaborative 
\underline{W}ork).

The window in Figure 5.4 will pop up. First select a hostname or enter a new one. 

If the selected hostname is available, the window in Figure 5.5 will appear. You 
can select the parameters for a connection.

Depending on the configuration parameters in {\it covise.config} the execution model 
and the time-out are adjusted. Now one can change the time-out and the execution mode 
if other values than the standard are required. 

For execution mode rexec the user id and password on the remote computer has to
be entered. For execution mode rsh or ssh only the user id is needed

\begin{covimg}{AdvancedTopics}{AddHost}{Host Selection Window}{0.7}\end{covimg}
\begin{htmlonly}
Figure 5.4: Host Selection Window.
\vspace{0.5cm}
\end{htmlonly}


In the manual execution mode COVISE writes a message in the window saying how COVISE 
should be started on the remote computer. It looks like 

\begin{verbatim}
   start "crb 31005 129.69.29.12 1005" on 
   visit.rus.uni-stuttgart.de
\end{verbatim}


\begin{covimg}{AdvancedTopics}{ConnectParameters}{Set Connection Parameters}{0.7}\end{covimg}
\begin{htmlonly}
Figure 5.5: Set Connection Parameters
\vspace{0.5cm}
\end{htmlonly}

The collaboration partner has to type in the following command (which has to be 
provided to him by means such as phone, video conference or email):
\begin{verbatim}
   crb 31005 129.69.29.12 1005
\end{verbatim}
                                     
When the remote computer is added successfully the remote username and hostname will 
appear in the module browser list (see Figure 5.5). Here the option is used that hosts 
are shown colored.

In a multiuser session (CSCW \latexonly $>>$ \endlatexonly \begin{htmlonly} > >
\end{htmlonly} AddPartner) a Mapeditor will pop up on the remote workstation.

\begin{covimg}{AdvancedTopics}{RemoteBrowser}
           {Remote Computer in the Module Browser}{0.7}\end{covimg}
\begin{htmlonly}
Figure 5.6: Remote Computer in the Module Browser
\vspace{1cm}
\end{htmlonly}

\clearpage
\section{Starting a module on the remote computer}


When selecting the remote computer in the hosts list the categories and modules 
available on this computer will be offered. Clicking on a module it is started on the 
remote computer. This is indicated by the hostname in front of the module 
name (Figure 5.7), if the hosts are not colored.

                              
\begin{covimg}{AdvancedTopics}{RemoteIcon}{Icon  for a Remote Module}{0.7}\end{covimg}
\begin{htmlonly}
Figure 5.7: Icon  for a Remote Module
\vspace{0.5cm}
\end{htmlonly}

Next the module ports have to be connected and parameters adjusted. It does not make 
any difference whether modules are executed locally or on a remote computer.

When a map is saved (menu File \latexonly $>>$ \endlatexonly \begin{htmlonly} > > \end{htmlonly} Save) the information about the hostname is 
saved, too. When a map is loaded which was saved including remote modules one is 
asked to add the remote hosts first


\section{Collaborative Working}


CSCW \latexonly $>>$ \endlatexonly \begin{htmlonly} > > \end{htmlonly} AddPartner includes the remote host in the session and starts a Mapeditor 
on the remote machine. Except for the renderers all other modules are started on the 
computer which was selected in the hosts list. Renderer modules are started on all 
workstations.

The partner who initiated the session initially has the master role. He can load maps 
or start modules and connect them. He also controls the renderers. The slave partners 
can watch all actions of the master but all menu items besides the menu master and 
interaction in the Mapeditor are deactivated. This is indicated by grey text on the menu buttons and in
the modules. The slave partners can request the master role using the menu 
MasterControl \latexonly $>>$ \endlatexonly \begin{htmlonly} > > \end{htmlonly} Request
(or use the corresponing item "MasterRequest" of the Viewer Popup Menu in the Renderer);
thereupon a window (Figure 5.8) pops up on the master computer:

\begin{covimg}{AdvancedTopics}{MasterRequest}{Master Request Window}{0.7}\end{covimg}
\begin{htmlonly}
Figure 5.8: Master Request Window
\vspace{0.5cm}
\end{htmlonly}


The slave Renderers are synchronised with the master renderer which means that all 
manipulation actions like changing the camera position, zooming, selecting objects 
etc. are sent to the slave Renderers. As long as the master doesn't do anything in 
the Renderer the slave Renderers can be used independently. 

One can make his own mouse pointer visible for the partners by pressing the SHIFT key 
and moving the mouse. This functionality is called Telepointer. In all remote renderers 
the originating hostname appears at the position pointed at (Figure 5.9). This also works
for Renderer windows having different sizes as the position in 3D space is transmitted 
and not the 2D pixel coordinates. 

Figure 5.9 shows a snapshot from a collaborative session. The 
user pw\_te on host richard.vircinity uses the telepointer to show the other
user(s) the 
backflow zone in a channel. 

\begin{covimg}{AdvancedTopics}{Telepointer}{Telepointer in the Renderer}{0.7}\end{covimg}
\begin{htmlonly}
Figure 5.9: Telepointer in the Renderer
\vspace{0.5cm}
\end{htmlonly}


