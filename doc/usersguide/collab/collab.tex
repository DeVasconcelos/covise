
\begin{htmlonly}

\input{../../htmlinc}
\input{../../usersguide/defines}

\end{htmlonly}


%=============================================================
\startdocument
\chapter{COVISE CE (Collaborative Engineering)}
\label{CollabEngineering}
%=============================================================

\section{Architecture and Configuration}
\label{Architecture}

\subsection{Architecture Summary}
\label{Architecture_Summary }

%{\bf Note:}

{\bf Usage Hints:}
\vspace{0.5cm}

This short introduction has been prepared for users with basic experience in using
COVISE VR in a standalone environment only; for this user group it provides the necessary
background to extend the use of COVISE to Collaborative Engineering, i.e. COVISE CE. The information 
has been collected from the Tutorial, the first chapters of the (old) User's Guide, and other sources
to provide one reference chapter for these users. 
   
\vspace{0.5cm}

For collaborative working you can either use COVISE CE alone - as described in this document - or
you can extend a collaborative COVISE session to a complete (virtual) meeting, using  
{\bf \begin{htmlonly}N'S³\end{htmlonly}\latexonly N'S$^3$\endlatexonly (COVISE Conference
Room Interface)}.
The Conference Room Interface (optional feature, described in a separate document)
\begin{itemize}
\item has been developed in the framework of the \begin{htmlonly}N'S³\end{htmlonly} 
\latexonly N'S$^3$\endlatexonly (ENScube) project.
\item is based on Sametime (Sametime is a trademark of IBM Lotus Corporation)
\end{itemize} 
\vspace{0.5cm}

{\bf Architectural Concepts:}

For collaborative working with COVISE you should know the basic architectural concepts of COVISE.
After having read this chapter you will be familiar with:

\begin{itemize}
\item The Architecture of COVISE
\item how to prepare COVISE for a distributed or collaborative session
\end{itemize}

In COVISE it is possible to run modules on remote computers. This is also known as 
"Distributed Computing". By distributing modules across a network one can make use 
of remote resources for example of a compute server with more CPUs or memory than on a 
local workstation or PC. The COVISE session is controlled from the Mapeditor on the
local workstation. Remote hosts are included in the session via the menu item 
CSCW \latexonly $>>$ \endlatexonly \begin{htmlonly} > > \end{htmlonly} AddHost (CSCW = \underline{C}omputer
\underline{S}upported \underline{C}ollaborative \underline{W}ork).

In a multiuser session each participant has his own Mapeditor and Renderer. The 
session has to be initiated by one partner who adds the other partners to the 
session (menu item CSCW \latexonly $>>$ \endlatexonly \begin{htmlonly}> > \end{htmlonly} AddPartner). The initiating partner plays the master role, 
which means that he has the control over the Mapeditor and the Renderer. If he e.g. 
changes the camera position in the Renderer all other partner's cameras are synchronised 
with the master camera. The master role can be exchanged between partners. This way of 
working together in a multiuser session is also known as "Collaborative Working" or - for COVISE
applications - as "Collaborative Engineering" where COVISE is regarded as a "Shared Application" 
which is aware of the sharing.

As the hosts of the partners can also be used for distributed computing COVISE extends 
far beyond a "Shared Application" such as the ones based on X Windows sharing or a 
Windows application shared via Netmeeting.

The next sections provide background information on the COVISE architecture and explain how a 
distributed session (Distributed Computing) or a collaborative 
session (for Collaborative Engineering) is implemented.

\begin{longtable}{|p{14cm}|}
\hline
\newline
{\bf See also: Additional feature COVISE daemon "covised":}\newline
The COVISE daemon "covised" - included as a {\bf preversion} in Rel. 5.2 - 
provides a more general and more comfortable user interface for collaborative 
working than CSCW, using a concept of "rooms" (working groups - can be {\bf predefined}) 
like N'S³ (see {\bf 5.6 New Collaborative COVISE})\newline 
\\									
\hline
\end{longtable}


\subsection{Distributed Computing}
\label{distributed}

In {\bf COVISE} it is possible to run modules on remote computers. This is also 
known as {\it Distributed Computing}. By distributing modules across a network 
remote resources are used for example of a compute server with 
more CPUs or memory than o<Zwischenablage leer>n your local workstation or PC. The {\bf COVISE} 
session is controlled from the {\bf MapEditor} on the local workstation. 

	
\begin{covimg}{collab}{distr-a}{Distributed Session (Distributed
Computing)}{0.5}\end{covimg}
\begin{htmlonly}
Figure 5.1: Distributed Session (Distributed Computing)
\vspace{0.5cm}
\end{htmlonly} 
  
The previous figure shows the elements of an example for 
distributed working in {\bf COVISE}. The 
application consists of three modules: a module which reads in data (READ) 
a module which extracts a special feature (FILTER) and a module which displays 
the extracted data (RENDER). As the filter module consumes much CPU 
time and memory it will be started on a remote compute server, also the Reader because the data  
to be read in is on the remote machine. 
The first process started by {\bf COVISE} is the {\bf Controller} 
which in turn starts the user interface process {\bf MapEditor} and the data 
management process {\bf CRB}. As soon as another host is included in the 
session a {\bf CRB} is started on that computer. The read module is started on the 
local workstation, the filter module on the remote computer  and the 
renderer on the local workstation. The green arrows between the processes 
{\bf Controller}, {\bf MapEditor}, {\bf CRB} and the modules indicate TCP sockets, the blue arrows 
indicate shared memory access. 

When the module map is executed the {\bf Controller} sends a start message to 
the remote read module. The read module reads in the data file and creates a 
{\bf COVISE} data object (1) in shared memory and after processing tells the 
{\bf Controller}  that the module  has finished. The {\bf Controller} informs the filter module on the 
remote computer to start. The filter module asks its data management 
process ({\bf CRB}) for the data object (1). The filter module now reads that data object, computes 
something and puts the data object (2) into shared memory. It then tells the 
{\bf Controller} that it has finished. The {\bf Controller} informs the renderer module to 
start. The renderer asks the {\bf CRB} for object (2) and as this object is not 
available on the local workstations the {\bf CRB} transfers it from the compute 
server into the shared memory of the local workstation (2'). Now the renderer can 
access this object and display the data. 

\subsection{Collaborative Working}
\label{collaborative}

In a multiuser session each participant has it's own {\bf MapEditor} and {\bf Renderer}. 
The session has to be initiated by one partner who adds the other partners to 
the session. The initiating partner plays the 
master role, which means that he has the control over the {\bf MapEditor} and the 
{\bf Renderer}. If he e.g. changes the camera position in the {\bf Renderer} all other 
partner's cameras are synchronised with the master camera. The master role 
can be exchanged between partners. This way of working together in a 
multiuser session is also known as {\it Collaborative Working / Collaborative Engineering}.
The hosts of the partners can also be used for distributed computing.


\begin{covimg}{collab}{collab-a}{Collaborative Session (Collaborative
Engineering in COVISE CE)}{0.5}\end{covimg}
\begin{htmlonly}
Figure 5.2: Collaborative Session (Collaborative Engineering in COVISE CE))
\vspace{1cm}
\end{htmlonly}
	 
In a collaborative session (see figure) a user interface process ({\bf MapEditor}) 
and a {\bf Renderer} are started also on the remote machine. The {\bf Renderer} 
module is the only module which is started on all computers in a session.


%
\clearpage

\subsection{Preparing COVISE for a Distributed or a Collaborative Session}

\begin{longtable}{|p{14cm}|}
\hline
\newline
{\bf See also: Additional feature COVISE daemon "covised":}\newline
The COVISE daemon "covised" - included as a {\bf preversion} in Rel. 5.2 - 
provides a more general and more comfortable user interface for collaborative 
working than CSCW, using a concept of "rooms" (working groups - can be {\bf predefined}) 
like N'S³ (see {\bf 5.6 New Collaborative COVISE})\newline 
\\									
\hline
\end{longtable}

Every computer that will participate in a distributed or collaborative session should be included in the section
HostConfig in the file covise.config. For each host you have to specify the memory model for data exchange between
modules on the local machine, the execution mode and a timeout for TCP connections. 

\begin{verbatim}
HostConfig
{
    # Hostname MemoryModel ExecutionMode Timeout
    vista       shm        rexec       30
    visit       shm        rexec       30
}
\end{verbatim}

For workstations and PCs the memory model is shm which stands for shared memory.

The execution mode specifies the command which should be used to start the CRB on the remote computer. Possible
execution modes are:

\begin{itemize}
\item rexec
\item rsh
\item ssh
\item nqs
\item manual
\end{itemize}

For all execution modes besides manual one needs access to the account on the remote 
computer. For rexec one has to enter the hostname, the user id and the password on the 
remote machine (similar to logging in on the remote computer using telnet). rsh and ssh 
can only be used if they allow to log in without password specification (see man rsh and 
ssh for the files where allowed users are specified). nqs is not recommended, it can be 
used to put the CRB into a batch queue. Manual means that one has to start the CRB 
process manually on the remote machine. This can be useful for sessions across a 
firewall or if one doesn't have access to the remote account. In this case COVISE 
shows a command in the window where COVISE was started.

The time-out value specifies how many seconds a process will wait to be contacted by a 
new process that he initiated (e.g. the Controller waiting for a module). The default 
value is 5 seconds. For slow networks a time-out of 30 seconds is useful. For very 
slow networks even a higher value is recommended.


\subsection{COVISE across Firewalls}

As shown in Figure 5.2 COVISE uses TCP sockets for communication with remote hosts. 
A socket is defined by an IP address, a port number and the protocol (here tcp). COVISE 
port numbers start by default at 31000. One can configure the start number in the 
file covise.config using the keyword COVISE\_PORT in the section network:
 
\begin{verbatim}
Network
{
    COVISE_PORT     5000
}
\end{verbatim}

For collaborative or distributed sessions across firewalls the firewall has to allow 
tcp connections to ports in both directions starting with the number defined in 
covise.config. You need as many ports as modules started during the whole session +3 
for distributed sessions or + 4 for collaborative sessions (if you load several maps 
in a session each map needs new ports). Depending on the execution mode the ports for 
rexec, rsh or ssh have to be allowed. For the execution mode manual no extra port 
is required.

Note:

If you use IP forwarding from your firewall to your local computer you
have to make additional configurations.
Every host that wants to connect to your session has to know that you
are behind a firewall and use IP forwarding.
Therefore you can tell COVISE not to connect to your machine but to your
firewall instead. This is done by adding an IP\_ALIAS entry on every client side. 
Assume your IP is 192.168.0.15 and your firewall has the IP 133.168.226.234 
from the outside. Then you have to add

\begin{verbatim}
Network {
   IP_ALIAS    192.168.0.15    133.168.226.234
   #           <your IP>       <your firewall IP> =

}
\end{verbatim}

to the config file on every host you want to connect to.


\section{CSCW}
\label{CSCW}

\subsection{CSCW Summary}

After having read this section you will be familiar with:

\begin{itemize}
\item including a remote host or partner in the session
\item starting a module on the remote computer
\end{itemize}

\begin{longtable}{|p{14cm}|}
\hline
\newline
{\bf See also: Additional feature COVISE daemon "covised":}\newline
The COVISE daemon "covised" - included as a {\bf preversion} in Rel. 5.2 - 
provides a more general and more comfortable user interface for collaborative 
working than CSCW, using a concept of "rooms" (working groups - can be {\bf predefined}) 
like N'S³ (see {\bf 5.6 New Collaborative COVISE})\newline 
\\									
\hline
\end{longtable}

Notes:
\begin{itemize}
\item It is strongly recommended to use COVISE version 5.2 (or higher) on all
participating hosts. Otherwise you may have different sets of options and you may run into
compatibility problems (e.g. with changes in the implementation of data types).  
\item Set Mirrors etc: see Section 4, Mirroring
\item Open Conference Room: By using the COVISE Conference Room Interface (optional feature, based on Sametime - Sametime is a trademark of IBM Lotus
Corporation) you can extend a Collaborative COVISE Session to a complete meeting - see COVISE Conference
Room Interface (separate document).
\end{itemize}

\subsection{Including a remote host or partner in the session}

\begin{covimg}{collab}{HostsMenu}{Hosts(CSCW) Menu}{0.7}\end{covimg}
\begin{htmlonly}
Figure 5.3: Hosts(CSCW) Menu
\vspace{0.5cm}
\end{htmlonly}

Figure 5.3 shows the menu item for adding a remote host or including a new partner 
into the session (CSCW = \underline{C}omputer \underline{S}upported \underline{C}ollaborative 
\underline{W}ork).

The window in Figure 5.4 will pop up. First select a hostname or enter a new one. 
.gif
If the selected hostname is available, the window in Figure 5.5 will appear. You 
can select the parameters for a connection.

Depending on the configuration parameters in {\it covise.config} the execution model 
and the time-out are adjusted. Now one can change the time-out and the execution mode 
if other values than the standard are required. 

For execution mode rexec the user id and password on the remote computer has to
be entered. For execution mode rsh or ssh only the user id is needed


In the manual execution mode COVISE writes a message in the window saying how COVISE 
should be started on the remote computer. It looks like 

\begin{verbatim}
   start "crb 31005 129.69.29.12 1005" on 
   visit.rus.uni-stuttgart.de
\end{verbatim}


\begin{covimg}{collab}{ConnectParameters}{Set Connection Parameters}{0.7}\end{covimg}
\begin{htmlonly}
Figure 5.5: Set Connection Parameters
\vspace{0.5cm}
\end{htmlonly}

The collaboration partner has to type in the following command (which has to be 
provided to him by means such as phone, video conference or email):
\begin{verbatim}
   crb 31005 129.69.29.12 1005
\end{verbatim}

When the remote  host is successfully added, the remote username and hostname will appear 
in the list of hostnames of the {\bf Module Browser}.  Depending on the specification 
in the scope {\it UIConfig} of the  config.file , the new host 
and his modules are shown either in another color or have the hostname in their label. 
Here the option is used that hosts are shown colored(see Figure 2.4).                                      
When the remote computer is added successfully the remote username and hostname will 
appear in the module browser list (see Figure 2.4). Here the option is used that hosts 
are shown colored.

\begin{covimg}{collab}{modulebrowser}
    {Module Browser Windows for Local and Distributed/Collaborative Working}{0.5}\end{covimg}
\begin{htmlonly}
Figure 5.6: Module Browser Windows for Local and Distributed/Collaborative Working
\vspace{1cm}
\end{htmlonly}

In a multiuser session (CSCW \latexonly $>>$ \endlatexonly \begin{htmlonly} > >
\end{htmlonly} AddPartner) a Mapeditor will pop up on the remote workstation.

\begin{covimg}{collab}{canvas_l_d}
           {Module Map for Local and Distributed/Collaborative Working}{0.5}\end{covimg}
\begin{htmlonly}
Figure 5.7: Module Map for Local and Distributed/Collaborative Working
\vspace{1cm}
\end{htmlonly}
\clearpage


\subsection{Starting a module on the remote computer}


When selecting the remote computer in the hosts list the categories and modules 
available on this computer will be offered. Clicking on a module it is started on the 
remote computer. This is indicated by the hostname in front of the module 
name (Figure 2.5), if the hosts are not colored.

                              
\begin{covimg}{collab}{RemoteIcon}{Icon  for a Remote Module}{0.7}\end{covimg}
\begin{htmlonly}
Figure 5.8: Icon  for a Remote Module
\vspace{0.5cm}
\end{htmlonly}

Next the module ports have to be connected and parameters adjusted. It does not make 
any difference whether modules are executed locally or on a remote computer.

When a map is saved (menu File \latexonly $>>$ \endlatexonly \begin{htmlonly} > > \end{htmlonly} Save) the information about the hostname is 
saved, too. When a map is loaded which was saved including remote modules one is 
asked to add the remote hosts first



\section{MasterCtrl}
\label{Master Control}

\subsection{MasterCtrl Summary}

After having read this chapter you will be familiar with:

\begin{itemize}
\item sending a Master Request (using MasterCtrl in the MapEditor or the Renderer PopUp Menu)
\item synchronization of Renderers
\item using the Telepointer
\item using the Chat Line
\end{itemize}

\subsection{Master Request}

CSCW \latexonly $>>$ \endlatexonly \begin{htmlonly} > > \end{htmlonly} AddPartner includes the remote host in the session and starts a Mapeditor 
on the remote machine. Except for the renderers all other modules are started on the 
computer which was selected in the hosts list. Renderer modules are started on all 
workstations.

The partner who initiated the session initially has the master role. He can load maps 
or start modules and connect them. He also controls the renderers. The slave partners 
can watch all actions of the master but all menu items besides the menu master and 
interaction in the Mapeditor are deactivated. This is indicated by grey text on the menu buttons and in
the modules. The slave partners can request the master role using the menu 
MasterControl \latexonly $>>$ \endlatexonly \begin{htmlonly} > > \end{htmlonly} Request
(or use the corresponing item "MasterRequest" of the Viewer Popup Menu in the Renderer).
Fig. 5.9 and 5.10 below show both possibilities.

\begin{covimg}{collab}{MasterCtrl}{MasterControl from Mapeditor}{0.7}\end{covimg}
\begin{htmlonly}
Figure 5.9: MasterControl from MapEditor
\vspace{0.5cm}
\end{htmlonly}

\begin{covimg}{collab}{MasterRequest}{MasterRequest from Renderer}{0.7}\end{covimg}
\begin{htmlonly}
Figure 5.10: MasterRequest from Renderer
\vspace{0.5cm}
\end{htmlonly}
\clearpage

If you click on MasterCtrl, a question dialog (Figure 5.11) pops up on the master computer:

\begin{covimg}{collab}{Request}{Master Request Window}{0.7}\end{covimg}
\begin{htmlonly}
Figure 5.11: Master Request Window
\vspace{0.5cm}
\end{htmlonly}

If the master replies {\bf No} you will be informed by the window shown in Fig. 5.12

\begin{covimg}{collab}{RequestDenied}{Negative Response}{0.7}\end{covimg}
\begin{htmlonly}
Figure 5.12: Negative Response
\vspace{0.5cm}
\end{htmlonly}

If the master replies {\bf Yes}
\begin{itemize}
\item all previously grey menu items on the former slave side are now black (selectable), only MasterCtrl becomes grey (not
selectable)
\item all previously black menu items on the master side are now grey (not selectable), only MasterCtrl becomes black
(selectable)
\end{itemize}

This applies not only to the Mapeditor menu but also to the parameter entries
in Module Info and Control Panel window - but please note the following
{\bf restrictions for scalar/slider parameters} if their appearance type has been changed
to "player"/"sequencer":
\begin{itemize}
\item A player/sequencer will be stopped if you transfer control to the slave
\item the "delta value" is will not be transferred to the slave; so if you don't like the
default value you have to explicitly set delta at the slave.
\end{itemize}
\clearpage

\subsection{Synchronization}

The slave Renderers are synchronised with the master renderer which means that all 
manipulation actions like changing the camera position, zooming, selecting objects 
etc. are sent to the slave Renderers. As long as the master doesn't do anything in 
the Renderer the slave Renderers can be used independently. 

Depending on the line speed of your connection, you can choose your adequate level of synchronization (see Fig.
5.13); use
LOOSE coupling if you have a slow connection.
  
\begin{covimg}{collab}{Sync}{Increasing levels of Synchronization}{0.7}\end{covimg}
\begin{htmlonly}
Figure 5.13: Increasing levels of Synchronization 
\vspace{0.5cm}
\end{htmlonly}

This is a summary of the different levels of coupling: 
\begin{itemize}
\item {\bf Master/Slave Coupling - Sync Field=SYNC (default value):}
Normally a master/slave relationship has been established between the partners of a working session. 
Only the master can change e. g. the camera position in his renderer, and the cameras of all other 
partners are synchronized with the master camera. In general, manipulators and editors become 
detached and invisible, the menu bar is set to insensitive in the slave renderers, and the scene in 
all slave renderers is updated according to the view in the master renderer. 
(Exception: Light information is not sent to other renderers in a cooperative environment ) 
The Sync field of the information area is set to {\bf SYNC}.
\item {\bf Tight Coupling - Sync Field=TIGHT:}
The only difference between TIGHT and Master/Slave (SYNC) is that 
TIGHT enforces permanent synchronization whereas Master/Slave updates the slave renderers at 
endpoints only.
The Sync field of the information area is set to {\bf TIGHT}.
\item {\bf Loose Coupling - Sync Field=LOOSE:}
If the master enables this mode all partners have the same interaction facilities as the master; 
the camera positions etc. are no longer sent to other renderers. The Sync field of the information 
area is set to {\bf LOOSE}.
\end{itemize}
\clearpage

\subsection{Telepointer}

One can make his own mouse pointer visible for the partners by pressing the SHIFT key 
and moving the mouse. This functionality is called Telepointer. In all remote renderers 
the originating hostname appears at the position pointed at (Figure 3.5). This also works
for Renderer windows having different sizes as the position in 3D space is transmitted 
and not the 2D pixel coordinates. 

Figure 5.14 shows a snapshot from a collaborative session. The 
user pw\_te on host richard.visenso.de uses the telepointer to show the other user(s) the 
backflow zone in a channel. 

\begin{covimg}{collab}{Telepointer}{Telepointer in the Renderer}{0.7}\end{covimg}
\begin{htmlonly}
Figure 5.14: Telepointer in the Renderer
\vspace{0.5cm}
\end{htmlonly}
\clearpage

\subsection{Chat Line}

\begin{covimg}{collab}{ChatLine}{Message Area and Chat Line}{0.7}\end{covimg}
\begin{htmlonly}
Figure 5.15: Message Area and Chat Line
\vspace{0.5cm}
\end{htmlonly}

The {\bf Message Area} is a scrollable text output window (lower part of Map Editor window) that shows warning and
information text produced by modules during their execution. In addition, this Message Area can be used for receiving
information sent from other partners. The {\bf Chat Line} is an editable text field (below the Message Area) that can be used
for sending information to other partners joining a COVISE session. After pressing ENTER, the contents are sent to all other
message areas of the participating user interfaces. A beep happens on the receiving host, and the information text is
highlighted (reverse video mode). The layout of these messages is

\begin{verbatim}
>> hostname >> Text
\end{verbatim}

\section{Performance Considerations in a Cooperative Session}
\label{Performance}
Particularly for performance reasons, it is advantageous to know little
about what is going on behind the scenes when the mouse is moved around
in the viewer area, especially when working with the master renderer in
the environment.

\subsection{Updating the Telepointer}

The telepointer is operating in all directions. If you press the SHIFT
key on your keyboard, your machine's name will appear at your current
mouse position in the other renderer's drawing areas. To reposition the
telepointer to another position, release the SHIFT key, move the mouse
and press SHIFT again at the new position, or move the mouse while the
SHIFT key is still pressed. The difference here is, that in the second
case the new mouse position is sent over the network very often.

%\subsection{Updating the Head Tracker}
%
%In head tracking mode, the current tracking update rate can be adjusted
%by the thumbwheel in the viewer preference sheet. A lower update rate
%produces less network traffic. The default tracking rate is currently
%0.1sec. The ideal tracking rate depends heavily on the scene complexity
%and hardware platform used. Also note that the others partner's graphics
%hardware might be not able to render at your locally available graphics
%speed. If you want the head tracking to have only local effect, change
%to loose coupling in the sync menu of the renderer's menu bar. When you
%switch back to tight coupling later on, the screens become automatically
%updated and synchronized again.


    
\subsection{Direct Manipulation}

Normally, new positional information is only sent, when the master
releases the mouse button in the viewer area. 

\subsection{Using the Decoration Around the Viewer Area}

The same is true for the thumb wheels and the slider around the
viewer. When you release the mouse button information is sent over
the network. 

\subsection{Using Sliders}

As far as the sliders in the transform editor are concerned, the situation
is somewhat different. If you are using the transform sliders by pressing
the mouse button and moving back and forth, every little movement will
directly go over the network. If you want to avoid this, click on the
slider once at the desired position or use the input line. Note that
renderers running on machines without advanced graphics hardware can
manually change the scene drawstyle changed to wireframe or points for
locally doing extensive editing operations. This especially applies to
the master/slave mode.


\section{Mirroring}
\label{Mirroring}

\subsection{Mirroring Summary}

After reading this chapter you will be able to work with a "mirror" of your map.

{\bf The main purpose of mirroring is to show a map in a slave without copying huge amounts
of input data.}
 
COVISE allows you to "mirror" a whole pipeline (also distributed) or parts of it completely to 
another host (and to delete it):
\begin{itemize}
\item {\bf Set Mirrors} - Sets the mirror for each host in the session
\item {\bf Mirror Node} - Mirror nodes (modules) of the current map to the chosen mirror host
\item {\bf Delete Mirrored Nodes} - Delete all mirrored nodes (modules); the original map remains unchanged
\end{itemize}
You can set new mirrors and add more hosts (see next 3 sections with a simple - not
realistic - example).
For a motivation to use mirroring see the realistic example in the last section.


\subsection{Set Mirrors}

In order to {\bf Set Mirrors}, you should have added at least one partner/host; in case of the example below
you are working as user "covise@sgi002.vircinity.de" and you have added "pw\_te@sgi001.vircinity.de" as a partner. You can
specify now sgi001.vircinity.de as mirror of sgi002.vircinity.de (and sgi002.vircinity.de as mirror of
sgi001.vircinity.de).
 
\begin{covimg}{collab}{setmirror}{Set Mirrors}{0.7}\end{covimg}
\begin{htmlonly}
Figure 5.16: Set Mirrors
\vspace{0.5cm}
\end{htmlonly}

\clearpage

\subsection{Mirror Nodes}

If you have set mirrors, you can issue {\bf Mirror Nodes} in order to mirror either a complete map (default) or a selected group of
nodes.


\begin{covimg}{collab}{markmirrornodes}{Mark nodes to be mirrored (default: complete map)}{0.7}\end{covimg}
\begin{htmlonly}
Figure 5.17: Mark nodes to be mirrored (default: complete map)
\vspace{0.5cm}
\end{htmlonly}

\clearpage

Now you can mirror the selected nodes (shown in yellow) and compare e.g. execution times of original and mirrored
modules. In the example below you see that the mirrored modules run faster, so you might choose to change
your map accordingly.

\begin{covimg}{collab}{mirrornodes}{Mirror nodes (and compare execution times)}{0.7}\end{covimg}
\begin{htmlonly}
Figure 5.18: Mirror Nodes (and compare execution times)
\vspace{0.5cm}
\end{htmlonly}
\clearpage

\subsection{Delete Mirrored Nodes}

{\bf Delete Mirrored Nodes} is just the inverse operation of {\bf Mirror Node}, i.e. it removes the mirrored nodes
and leaves your original map unchanged.

\subsection{Example}

The advantage of mirroring modules to your partner's host is that you save the time of transferring the whole visualization
data every time the object changes. Changing the camera position in the renderer doesn't change the visualization data itself.
Therefore if you only use functions of the Renderer, the visualization data will not be transferred multiple times. You don't need the 
Mirror mode feature in this case.
But if you really want to change the visualization data during the session you should use this mode. For example if you want to change the 
data mapped on the surface from temperature to pressure you can take advantage of the mirror mode.

Suppose you want to share the visualization of a crash with your partner. The data has 53 time steps and the whole data size
including all these time steps is 20MB. It was computed with 54.800 elements. Assume
you use an ISDN line and can transfer 8kB/s data in average. Then it takes around 43 minutes to transfer the data to your partner.

You are in the session now and you want perhaps just change the color map of your visualization to analyse a certain
range of values. Do you really want to wait 43 minutes before you can continue the meeting?

The mirror mode solves this problem in the following way: You have to tranfer once your data to your partner's side. 
Every execution you do on your side is also done on your partner's machine. Therefore no visualization data has to be transferred.
All changes are updated on both sides independently. You can see the status of execution on your partner's side in the Mapeditor.

{\b NOTE:} The path- and filenames of your data have to be identical on every machine! The easiest way to create this scenario
is to put the data in your COVISE directory. All modules of COVISE allow the usage of path names that are relative the current COVISE directory. 
  

\clearpage

\section{New Collaborative COVISE (Daemon)}
\label{NewCollab}

This chapter introduces a new concept of Collaborative COVISE that is more general and
more flexible than using the CSCW operations AddPartner etc.

COVISE Rel. 5.2 provides a first implementation of this approach - 

- {\bf use this preversion at your own risk}.  

Section 1 is a concept paper about this approach that provides the necessary background
even if it might be obsolete in some details.

Section 2 shows you the implementation (your actual view might be slightly different).

Section 3 provides you the actual information to use the COVISE daemon "covised".


\subsection{Concept of Collaborative COVISE}

\subsubsection{What's new?}


The goal of this project is to improve the way of using COVISE in collaborative mode 
and provides the following features:
\begin{itemize}
\item every user has the information about his potential partners and is able to ask 
to be accepted into a certain group. Also, if he wants to show his 
work to the others, he is able to create a new group (we name it "room") 
where he is the master. In this case, the others can join or he can invite some 
partners or "force" the hosts to  work together.
\item the connection between partners is done once; every COVISE session has 
already this information and adds automatically the partners or hosts;
\item the concept is able to handle firewalls.
\end{itemize}

\subsubsection{Structure and Functionality}

The management of the participants and conference rooms is done by a special 
application named {\bf "covised"} (COVISE daemon). It can launched by every user and can be 
configured to run with or without user interface. Every daemon reads from a 
configuration file (.covised) a list of users he is interested in (masters list) and 
a list of predefined rooms. A room is a list of potential partners and is directly 
related to the user who created it (the master of the room) . It can be 
private (only the name of the room is propagated to the members ), normal 
(the members are propagated but only inside the room - only this version is currently
implemented) or public (the name of the members and the name of the room are propagated to everybody who asks).

\begin{verbatim}
Example of .covised file for user_x@host_j:

#Masters_list  
  user1@host1
  user2@host2
   ...

#Rooms_list
Room1_user_x{
  user3@host3
  user4@host4
  user4@computeServer  host  "ssh hwwo2k 'covised -host user_x@host_j'" 
  ...
 }
\end{verbatim}
 
When the covised is launched it tries to establish first the connections with every 
user of his masters list. Some of them are already connected and ready to inform him 
about the rooms where they take part (entire list of potential participants and 
those who already work together with a different color). If there are some predefined 
rooms in his list (he is the master), the application will try to connect to all 
potential members of the room(s) and if they run covised and he is included in their 
masters list, a (info) connection will be established and the information about rooms 
will be sent. 

\begin{covimg}{collab}{concept}{Concept of Collaborative COVISE}{0.7}\end{covimg}
\begin{htmlonly}
Figure 5.19: Concept of Collaborative COVISE
\vspace{0.5cm}
\end{htmlonly}


This is the first step: obtaining the informations about rooms and partners, and it 
is done automatically by the application opening some permanent connections between 
the members who have launched covised ("passive members"). That does not mean that they 
already work together: they need to be invited or to ask to be accepted to become 
an "active member" of a certain room. In the figure above, the user1@host1 which 
is not the master of the Room1\_user\_x is also displayed with different color because 
he was invited  by the user1, he accepted the invitation and he is considered now 
an active member (and the option "Join a room" will be transformed 
into "Invite a partner" for the same list). When the user\_x wants to join the 
Room1\_user\_1 (double click on the room), a request will be sent to the master, which 
will accept (send a message to all connected members of the room:  user6, user\_x) 
or will refuse the request (sending a message to user\_x). 

A special situation is with the user5 who does not not have an user interface. 
In this case he will try to connect  to  user1 (read from .covised) and will 
become automatically an active member of the rooms which are defined by user1 
(Room\_1\_user\_1) and include him as member. 

The non-GUI covised will also be used if there is a firewall  between host1 and 
host5, for example, and host5 can be used as "host". user1 opens a secure 
shell (ssh) on host5, launches covised without GUI but with an argument "-host". 
The connection between daemons will be done and a COVISE session will add 
automatically user5 as "host". If the host is defined in the .covised file, the 
daemon could also be started automatically by a command in the file as shown in 
the example file for user4@computeServer.

But the final goal of the users is to have a collaborative COVISE session. It will 
be launched by the master (does not matter the number of active members), which will 
press a button attached to the room. The controller, crb and MapEditor will be 
launched and if there are already some active members, a crb command will be 
generated (as in manual "AddPartner") and will be transmitted through the directly 
opened connection to their covised, which will launch the crb, and everything will 
work as in old collaborative covise. As soon as there are new active members 
they will be added automatically as partners.

An active member can quit by asking the master or can be eliminated by him. 
An automatic "remove partner" will be done in the COVISE session, but the connection will 
be kept because he is still a passive member. An exit from collab\_covise will also close 
the connection. For detecting crashes,  there will be  an exchange 
of messages between the master and slaves every 30s.

\subsubsection{Multiuser Concept}   
 
If there is more than one user on the same machine running covised, the usage 
of a fix port (31000) is not sufficient. The first process, which accepts connections 
on 31000, will route, in this case all, messages between other users and their 
partners. If a new user on the same machine launches covised, it will connect to 
31000 for sending or receiving messages to or from outside through the first user.
 
If the first user terminates covised or crashes, his role should be taken by another 
daemon of another user. For doing that, there will be a priority list (FIFO principle) 
which will be sent to all local users.  When the "owner" of 31000 will exit, the 
first in the list will listen to 31000 and the others will connect there.   

\clearpage

\subsection{Implementation}

The pictures below show you an example of the implementation of Collaborative
COVISE with "covised" (Your actual view might slightly differ from the images shown below).


There are 2 members
\begin{verbatim}
 covise@sgi001 
 covise@sgi001
\end{verbatim}
and the following configuration of rooms
\begin{verbatim}
active:     Room 2 - Master: active members covise@sgi001 and covise@sgi002

potential: (Room 1 - Master) 
            Room 3 - Master: potential member covise@sgi001 
(see Fig. 5.20)

active:     Room 2  - Slave: active members covise@sgi001 and covise@sgi002  
           (Room 2X - Master)    
potential: (Room 1X - Master) 
            Room 3X - Master: potential members covise@sgi002 and covise@sgi001
(see Fig. 5.21) 
\end{verbatim}

\begin{covimg}{collab}{implementation1}{active: Room 2 - Master / potential: (Room 1 - Master) Room 3 - Master}{0.7}\end{covimg}
\begin{htmlonly}
Figure 5.20: \newline
active:     Room 2 - Master \newline
potential: (Room 1 - Master) Room 3 - Master   
\vspace{0.5cm}
\end{htmlonly}

\begin{covimg}{collab}{implementation2}{active: Room 2  - Slave (Room 2X - Master) / potential: (Room 1X -
Master) Room 3X - Master}{0.7}\end{covimg}
\begin{htmlonly}
Figure 5.21: \newline
active:     Room 2  - Slave  (Room 2X - Master) \newline   
potential: (Room 1X - Master) Room 3X - Master   
\vspace{0.5cm}
\end{htmlonly}

\clearpage

\subsection{Instructions for using the COVISE daemon}

The following section provides you the necessary information how to use the COVISE Daemon ("covised")
as a comfortable and more powerful replacement of the CSCW menu in the MapEditor.

\subsubsection{Introduction to the COVISE daemon ("covised")}

\begin{itemize}

\item Function: \newline
	The COVISE Daemon "covised" is a special software which allows the creation of
        working groups of COVISE users in collaborative mode.

\item Terminology:
    
    \begin{itemize}
	\item 	The working groups are called "rooms". They are used for
              	sharing information within a COVISE session.
	\item 	The user who has created/predefined a working group is called
              	"master" of this room (working group)
	\item 	Any user who is in the list of available partners for
              	a defined room is called "member" of this room.
 	\item   Rooms can have two type of members, displayed in separate lists:
	
	 	\begin{itemize}
	 		\item 	"Active members" are the members which have joined
                       		the room in order to participate in a COVISE session.
              		\item 	"Potential members" are the members which have started a "covised"
                       		session but have not yet joined the room.	
         	\end{itemize}
		
 	\item 	When launching a "COVISE" session in one of the defined rooms,
          	the main part of "COVISE" is started for the user which is
          	the "master" of the room. For the other users from the list
          	specified for that room,  "COVISE" is started in "partner"|"host"
          	mode.
     \end{itemize}
     
\end{itemize}
     
\subsubsection{Using the COVISE daemon ("covised")}

In order to use the COVISE daemon "covised" the following operations are provided:
   
\begin{enumerate}

\item Prepare the environment for using "covised":\newline
      Before starting "covised", an initialization file can be prepared. 
      The name of the file has the following format: ".covised[-hostname]",
      where "hostname" is optional.\newline
      The initialization file has two sections :
      	\begin{itemize}
        	\item 	\#MASTERS\_LIST section with the names of the users and
        		hosts to which covised tries to establish connection
         	\item   \#ROOMS\_LIST section which contains predefined
        		working groups established for using COVISE in collaborative mode.
	\end{itemize}
	
\item Start a "covised" session:
        \begin{itemize}
		\item 	After the start, "covised" tries to establish connections to the
        		COVISE daemons started by the other users specified in the \#MASTERS\_LIST section.
        	\item 	After connecting to the users from the "\#MASTERS\_LIST", covised
        		tries to establish connections to the members of the rooms defined in the
        		initialization file but not yet connected.
        	\item	If the connection is established the process of exchanging the available
        		"rooms" is started in both directions. The "rooms" are sent only to
        		the users which are members of those "rooms".
	\end{itemize}

\item   Create active rooms:
        \begin{itemize}
		\item 	After covised has been started the available rooms including the members 
			to which connection has been established are displayed in the
        		room list. The rooms created by the user who has started "covised"
        	 	have a "M" icon on the left side of the room name. The rooms received
        		by connected partners don't have this icon.
        	\item  	If the master of the room joins a room by clicking on "Join",
        		the room becomes an "active" room and is displayed on the
        		upper part of the covised user interface. Only the active rooms are
        		sent to the partners.
	\end{itemize}

\item   Start a COVISE session:
	\begin{itemize}
        	\item  	Click on "Launch": \newline
			A COVISE session is started including
        		all the active members of the room.
	\end{itemize}
	
\item   Invite a partner:
	\begin{itemize}
        	\item  	Select a potential member of the room and click on "Invite": \newline
			The selected member becomes an active member of the
        		room. If a COVISE session has already been started the selected
        		member is included in the session.
	\end{itemize}
	
\item   Remove a partner:
        \begin{itemize}
		\item 	Select an active member of the room and click on the "Remove": \newline
			The selected member becomes a passive member of
          		the room. If a COVISE session has already been started 
			the selected member is removed from the session.
	
	\end{itemize}
\end{enumerate}
	
        		

\clearpage

\section{Web Interface}
\label{WebInterface}

{\bf Please note: This feature is included as a preversion and with draft documentation - use at your own risk!}

  The web\_interface allows a web user to display in his browser
  the current renderer view from a running covise session.
  In order to do that execute the following steps:
  
  \begin{enumerate}
  \item Set the configuration file "init.www" : \newline
        This file contains the next environment variables which have to be set :
	\begin{itemize}
         \item HOSTSRV - the name of the host where the web server is running
         \item SERVER\_PATH( optional) - the path where the files needed to run the web interface are stored
         \item HTTP\_PORT - the number of the port on which the web interface is accessed from a web browser
         \item COVISE\_PORT - the number of the port on which the web interface is accessed
	 from the COVISE session
        \end{itemize}
	
	{\LARGE IMPORTANT:}
	
	\begin{longtable}{|p{13cm}|}
        \hline
	        Before starting the web\_srv, a COVISE session, or
        stopping the web\_srv, the environment variables
        have to be set using command "source init.www"\\
	\hline
        \end{longtable}
        

  \item Start the web\_interface :
        \begin{itemize}
         \item execute command web\_srv
         \item if the following error is displayed
        \begin{verbatim}
	       ERROR: bind to port : xxx ... 
	\end{verbatim}
        (where xxx is the port number of the HTTP\_PORT or COVISE\_PORT)
        the corresponding port (HTTP\_PORT or COVISE\_PORT)
        should be changed to a new value. Set the corresponding environment
        variable to the new value and restart the web\_interface.
        \end{itemize}
	
  \item Access the web\_interface :
        \begin{itemize}
	\item Using a a browser with a VRML plug-in installed (for example CosmoPlayer) the user should access the following URL address:
        \begin{verbatim}  
	       http://HOSTSRV:HTTP\PORT/ClientApplet.html
	\end{verbatim} 
        \item The web page displayed will contain a display area for VRML objects
          and a list of "registered renderers". This list contains the
          identifiers of VRML renderers connected from covise sessions.
        \item The user can select one of the available VRML\_renderers, and by using
          Connect button the objects stored in the VRML renderer are displayed
          in the browser.
        \item By selecting the "dynamic synchronization" option, the changes of the
          objects from the selected VRML renderer are also shown in the browser.
          If the user wants to interact directly with the objects
          displayed in the browser then he should deselect this option.
          To restore the synchronization with the COVISE Renderer the COVISE
          viewpoint should be reselected and the synchronization option
          reactivated. (After interacting with the objects in the
          browser area the name of the viewpoint is displayed with different fonts)
        \end{itemize}
 
 \item Use COVISE with the web\_interface;
         \begin{itemize}     
         \item In the COVISE session the objects you want to display in the browser
         have to be connected to the VRML renderer module.
         \item For interaction with the objects the IVRenderer module has to
         be used with the same input objects connected to the VRML renderer.
         If the web user allows synchronization with the renderer
         the changes made by user in the IVRenderer are transferred to the
         web browser.
         \item In the COVISE session one pair of the IVRenderer/VRML\_renderer modules can
         be used. If a new map is opened in covise session
         the identifier of the new VRML\_renderer is sent to the browser.
         \end{itemize}
	 
  \item Stop the web\_interface:
         \begin{itemize}
         \item execute command 
	 \begin{verbatim}
	        tsc_client
	 \end{verbatim}
	 \end{itemize}
\end{enumerate}





