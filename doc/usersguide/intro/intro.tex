
\begin{htmlonly}
 
\input{../../htmlinc}
\input{../../usersguide/defines}

\end{htmlonly}


%=============================================================
\startdocument
\chapter{Introduction}
\label{Introduction}
%=============================================================

\covise\ stands for {\bf CO}laborative {\bf VI}sualization and {\bf S}imulation {\bf E}nvironment. 
It is an extendable distributed software environment to integrate simulations, postprocessing and 
visualization functionalities in a seamless manner. From the beginning \covise\ was designed for 
collaborative working, allowing engineers and
scientists to spread on a network infrastructure. Processing steps can be arbitrarily distributed across 
different machine platforms to make optimal use of their varying characteristics. High 
speed network architectures of different kinds can be properly incorporated into 
\covise. Industrial or research simulation codes are easily integrated into this distributed 
software environment by wrapping the code as a
\covise\ module. If required, the open design allows easy extension of the \covise\ architecture. 
    
None of the currently available visualization packages supports all of the following features 
availabe in \covise.

\begin{itemize}
\item Distributed Working

\item Supercomputer Usage (parallel or vectorized)

\item Collaborative Working

\item Integration of Custom Codes

\item Virtual Reality 

\item Time Dependent Simulation
\end{itemize}

\begin{covimg}{intro}{gesamt}{Visualization of Data with \covise}{0.35}\end{covimg}


\covise\ is a modular and object-oriented software system. To visualize data 
several processing steps, called modules, are used. Each module is executed as a 
seperate operating system process and communicates with the 
central {\bf Controller} and the local data request broker {\bf CRB} by sending or 
receiving control messages via TCP/IP sockets. 
Modules are connected into a strictly unidirectional data flow 
network also called module map. Loops are not
allowed, nevertheless, there are possibilities to send feedback messages from
later to earlier modules in the processing chain. A Qt based user interface 
the \mapeditor\ gives the user a tool to perform all the necessary interactions.

Data exchange is handled different from control flow. Within a machine
pointers to shared memory are used to avoid copying of data objects. Between
machines data objects are transferred by the \covise\ request broker 
{\bf CRB}, including necessary format conversions.  


\section{How \covise\ works}

Most visualization systems currently available focus on the visual programming 
paradigm in an algorithm-oriented way. Data itself cannot be accessed by the user 
directly, but exists only internally. Means are provided to connect modules to networks 
which perform certain visualization tasks, but the access to the underlying 
data mostly is limited to typing a filename in the input module. 

There is no explicit control of data by users within most of the current dataflow 
based visualization systems. Thus either data produced by intermediate steps is 
kept even if it is not needed any more, or this caching mechanism can be switched 
off globally. As data does not exist as directly accessible data objects, selective 
handling is not possible. On the other hand a user who wants to examine a certain 
time interval repeatedly would be delayed by the application creating the same 
temporary objects over and over again instead of creating the sequence once and 
then displaying it just from cache.

Based on experience with own developments 
other packages available commercially or as open source, a system architecture has been
designed which fits the needs of a high performance distributed visualization application. 


\begin{covimg}{intro}{local_a}{Local Working in \covise}{0.5}\end{covimg}
    
A modular approach allows for the most flexibility in distributing certain 
parts of the visualization application to specific computers. The need for excellent 
high speed network utilization makes it necessary to put emphasis on the management 
of the network connections depending on the nature of transferred data.
The database approach makes a data request broker {\bf CRB} necessary.
This combination defines the \covise\ architecture.

The {\bf Controller} is the central part of this architecture. It has the overall view of the 
whole application. This  {\bf Controller} supervises the distribution of modules 
across the involved computers as well as the management of the execution of the 
application.


So an application module only needs 
connections to the {\bf Controller} and the request broker {\bf CRB}. The {\bf Controller} supplies the 
application module with the information that is necessary to guarantee the proper 
execution of the overall application. The data that will be exchanged between subsequent 
application modules is stored under the control of the request 
broker {\bf CRB}. This allows a very simple structuring of an application module.

The implementation of the \covise\ system architecture is done in C++. The basic 
communication functionality is provided as a library. 

For Distributed and Collaborative Working see Chapter 5, COVISE CE.


\section{History}

 The development of \covise\ was initiated in 1993 in the CEC RACE project R2031 named PAGEIN (Pilot
 Applications in a Gigabit European Integrated Network). The aim of PAGEIN was to evaluate possibilities of
 distributed computing and collaborative engineering on top of European high speed network infrastructures.
 One of the activities was the design and development of a software architecture as a testbed for the
 evaluation. The design of this basic architecture was led by the Visualization Department of the University of
 Stuttgart Computing Center (RUS). It was later called \covise. Also the main components of \covise\ as well
 as many application modules have been developed at RUS. With the project partners group from the
 aerospace field the application scenario was the simulation and analysis of air flow in the design phase of
 new airplanes. While initially industrial partners only defined their requirements they became more involved
 when they recognized the potential of CSCW (computer supported cooperative working) and \covise\ for the
 engineering field. 

 As a result \covise\ was used in the Esprit project ADONNIS (E9033) between Daimler-Benz Aerospace
 Airbus (DBAA, Bremen, Germany) and RUS (Stuttgart, Germany) via a 2 MBit/s leased line (permanent for
 one year). This allowed engineers of DBAA to evaluate cooperative working in an engineering simulation
 and design department. 

 In the project EFENDA sponsored by the German ministery of education and research BMBF the integration
 of modules from the airplane design field into \covise\ as homogeneous software integration platform was
 performed to increase the productivity of the airplane developer.  

 In the final phase of the ADONNIS project a short demonstration of CSCW applied to the analysis of
 vibration simulations of satelites was given. This prove of concept led to the definition of an Esprit best
 practice and demonstration project ACATAD with CASA (CONSTRUCCIONES AERONAUTICAS SA
 Division Espacioas) as the primary industrial partner in which collaborative analysis of dynamic simulations
 among satellite producer and sub contractors is introduced. \covise\ will be used across multiplexed ISDN
 lines. 

 Also the \covise\ development was initated in the aerospace field the underlying concepts and architecture
 are independent of a certain application field. Thus it was possible to also apply \covise\ to the automotive
 applications. 

 In the Esprit project (E20184) HPS-ICE, (High Performance Simulation of Internal Combustion Engines),
 INDEX (E22745), COVAS (E22542), the BMBF-project EFENDA, the G7-Projects SPOCK and GWAAT,
 the Collaborative Research Center (SFB374) and many other national projects. 

 Since 2004, the development of \covise\ is a joint effort by HLRS at the University of Stuttgart
 and RRZK at the University of Cologne.

 In 1997 the developers of \covise\ founded the company {\bf Vircinity IT-Consulting} 
 to bring \covise\ to the market. \covise\ is now distributed by VISENSO GmbH.

\section{First Steps}


On Windows, a new \covise\ session can be initiated from the Start menu or by Desktop icons.
On UNIX systems, the installation process appends the path to the \covise\ executables
and modules to the environment variable \$PATH.
Thus, starting \covise\ should be as easy as typing \verb/covise/ in a shell window.
If the command \verb/covise/ is not found, please contact your system administrator.

Initiating a \covise\ session will start the following processes:
\begin{enumerate}
\item {\bf Controller} (\verb/covise/)
\item {\bf CRB} (\verb/crb/)
\item \mapeditor\ (\verb/mapeditor/)
\end{enumerate}

After the starting phase the \mapeditor\ window will 
appear \latex{(see Fig.\ref{figmapeditor})}. 



\subsection{Start Parameter}

\covise\ can also be started with parameters. Typing  \verb/covise --help/  will show you the syntax. 

\vspace{0.5cm}

The following start options may be useful for you:
\begin{itemize}
\item {\bf -i}  start with \mapeditor\ as icon
\item {\bf -e}  execute immediately after loading
\item {\bf -q}  quit after 1st run ({\bf only together with -e})
\end{itemize}

\subsection{Configuration File}
\label{config}
\covise\ has a central configuration file \verb/covise.config/, which resides in
the \$COVISEDIR directory. The file consists of sections named {\it
scopes}, which look like :

\begin{verbatim}
Scope-name { : hostname}
{
	Var-Name1	string2
	Var-Name2	string2
}
\end{verbatim}


\subsubsection{HostConfig}
see also Chapter 5, COVISE CE)
Each computer that will participate in a distribited or collaborative session must be included in the
scope HostConfig. For each host the hostname, the memoy model, the execution mode and a timeout have to
be set .

\begin{verbatim}
HostConfig
{
# Hostname   Shared Memory     execution mode      timeout [s] Min. SHM
#           (shm|mmap|none) (rexec|rsh|ssh|manual) (default 5)  segment
   mike          shm               ssh               360          32MB
   peter         shm               manual            360
   george        shm               rsh               360
}
\end{verbatim}

For workstations and PCs the memory model is shm (shared memory).

When using shared memory, \covise\ manages multiple shared memory segments
and tries to put its data object in free spaces of these segments. If no
memory is left, it will allocate an additional segment. The size of this
segment is the minimum of the required ize for the object and the minimum
allocation size specified in the config file.

Small minimal SHM segment sizes will reduce memory consumption, but increase
the number of segments and add overhead. Both maximum size of shared memory
usage and number of segments are limited by operationg system and machine
configuration.

If no value is given, the following defaults are used:

\begin{verbatim}
Linux:              8 MB
SGI n32 and HP:    16 MB
SGI 64bit:         64 MB
\end{verbatim}

\subsubsection{UIConfig}
The user interface looks for the scope UIConfig. The variable {\it ShortCuts} contains the name
of favourite application module names. If the variable {\it ModuleIcons} is set to colored, module icons
for different host have different colors on the \mapeditor\ canvas.

\begin{verbatim}
UIConfig
{
ShortCuts  RWCovise Colors Collect CuttingSurface IsoSurface Renderer
ModulIcons colored
}
\end{verbatim}

In addition, you can use UIConfig to specify a browser for displaying online help or other
online documentation; default is  
\begin{verbatim}
Browser        netscape
\end{verbatim}
Please note that online help and documentation is optimized and tested with Netscape, so there
might be minor problems with other browsers.

\subsubsection {Additional information for using a MultiPC system:}

In order to improve the performance of COVER under Linux, you can use 2 synchronized PCs 
running in parallel instead of 1 PC using a dual graphic card. One of them will be the 
master and will be connected to the tracking system. The PCs will be connected through
TCP/IP and serial connection. The serial cable will be plugged into one
of two serial ports which has to be specified in cover.config, section MultiPC, 
key "Serial\_Port". Master and Slave (names of the machines) are defined in the same
section. In addition, you have to define the type of connection between the hosts in 
covise.config, section HostConfig (like in collaborative working).


Example:
\begin {verbatim}
HostConfig
{
#  Hostname    Shared Memory    execution mode    timeout in seconds

      pc1        shm               rsh               -1

      pc2        shm               rsh               -1
}

MultiPC
{
    Master        pc1
    Slave         pc2
    Serial_Port  /dev/ttyS0
}
\end{verbatim}

\subsubsection{License}
A very important scope in the configuration file is the license key. Without such a key no \covise\ 
can be started.

\begin{verbatim}
License
{
Key NFLHOODOLEBLILIEDEMLMNJGAJDPPHHHCDCIDPGDHABJKAKN    visage    31.12.2001 
}
\end{verbatim}


Most of the currently existing scopes are mainly used by the controller, the user interface, the
desktop renderer and the VR renderer. You can find more details about single scopes in the
chapters explaining these central \covise\ parts.



